\documentclass[german]{beamer}

\newcommand{\GENNF}{\textsc{gennf}}
\newcommand{\METACVS}{\textsc{metacvs}}
\newcommand{\CVS}{\textsc{cvs}}
\newcommand{\LISP}{\textsc{lisp}}
\newcommand{\COMMONLISP}{\textsc{common lisp}}
\newcommand{\LIBGPGME}{\textsc{libgpgme}}
\newcommand{\OSICAT}{\textsc{osicat}}
\newcommand{\PORTPATH}{\textsc{portpath}}

\usepackage{babel}
\usetheme{Rochester}
\usefonttheme{serif}
\usecolortheme[RGB={120,0,16}]{structure}

\setbeamertemplate{navigation symbols}{}

\author{Fabian~Otto~{\scriptsize (\texttt{sigsegv@cs.tu-berlin.de})}
  \and Hannes~Mehnert~{\scriptsize (\texttt{mehnert@cs.tu-berlin.de})}
  \and Florian~Lorenzen~{\scriptsize (\texttt{florenz@cs.tu-berlin.de})}}

\title{\GENNF}
\subtitle{Verteiltes Versionsmanagement mit Code Signierung}
\institute{TU~Berlin, FG Formale Modelle, Logik und Programmierung, \\
Infrastrukturen zur Open Source Softwareentwicklung WS05/06 \\
Bernd Mahr, Steffen Evers}
\date{31.~Januar~2006}


\begin{document}
\AtBeginSection[]
{
  \begin{frame}
    \frametitle{"Ubersicht}
    \tableofcontents[currentsection]
  \end{frame}
}

\AtBeginSubsection[]
{
  \begin{frame}
    \frametitle{"Ubersicht}
    \tableofcontents[currentsubsection]
  \end{frame}
}

\frame[plain]{\titlepage}

\begin{frame}
  \frametitle{"Ubersicht}
  \tableofcontents
\end{frame}

\section{Idee}

\begin{frame}
  \frametitle{Zielsetzung}
  \begin{itemize}
  \item Verifikation der Herkunft des verwalteten Codes
    \begin{itemize}
    \item Bisher mu\"s dem Repository server vertraut werden
    \item Code soll vertrauensw\"urdig sein, durch Signaturen von Entwicklern
    \end{itemize}
  \item Kein zentrales Repository:
    \begin{itemize}
    \item Anpassung der Repository- an die Arbeitsstruktur des Projektes
    \item Keine vollst"andige Replikation von Repositories $\rightarrow$
      Vernetzung durch Links
    \item Einfaches Verzweigen und Zusammenf"uhren von Entwicklungslinien
    \end{itemize}
  \item Gro"se Verf"ugbarkeit
  \end{itemize}
\end{frame}

\begin{frame}
%%  \frametitle{Umfeld/existierende L"osungen}
  \frametitle{CVS, darcs, GNU Arch, Mercurial, SVK, MCVS}
%   \begin{itemize}
%   \item
%   \end{itemize}
  \begin{block}{Verteilung}
    \begin{itemize}
    \item Verteilung
    \item Benutzung vorhandener Infrastruktur (SourceForge, Berlios,...)
    \item Portabilit"at
    \end{itemize}
  \end{block}
  \begin{block}{Signierung}
    \begin{itemize}
    \item Signierung von Sourcecode
    \item Buchpr"ufung von Sourcecode (Auditing)
    \end{itemize}
  \end{block}
  
%   \begin{block}{Verteilung}
%     \begin{itemize}
%     \item Zumeist vollst"andige Replikation des Repositories
%     \item Keine Vernetzung?
%     \item vorhandene Infrastrukturen nicht immer genutzt.
%     \end{itemize}
%   \end{block}
%   \begin{block}{Signierung}
%     \begin{itemize}
%     \item Darcs (mail)
%     \item GNU Arch (Repositories)
%     \end{itemize}
%   \end{block}
\end{frame}

\begin{frame}
  \frametitle{\METACVS}
  L"ost Problem der Verzeichnisversionierung in \CVS:
  \begin{itemize}
  \item Abspeicherung einer flachen Struktur (F-files)
  \item Verzeichnishierarchie wird "uber separate Abblidung
    gespeichert
  \item Clientseitige Erweiterung, ben"otigt \CVS{} als Backend
    $\rightarrow$ gute Verf"ugbarkeit
  \item \emph{Diese Ideen "ubernehmen wir}
  \end{itemize}
\end{frame}

\begin{frame}
  \frametitle{Planung}
  \begin{enumerate}
  \item Festlegung des Projektziels
  \item Konzeptionierung:
    \begin{itemize}
    \item Verteilung
    \item Code Signierung
    \end{itemize}
  \item Implementierung eines Prototypens
  \end{enumerate}
\end{frame}

\section{Konzept}

\begin{frame}
  \frametitle{Entwurf}
  \begin{itemize}
  \item Verteilung
    \begin{itemize}
    \item Abstraktes Modell f"ur verteilte Repositories
    \item Entwurf der Datenstrukturen und Operationen
    \item Implementierungshinweise
    \item Typische Anwendungsf"alle
    \end{itemize}
  \item Code Signierung
  \end{itemize}
\end{frame}

\subsection{Verteiltes Versionsmanagement}

\begin{frame}
  \frametitle{Strukturen}
  
\end{frame}

\begin{frame}
  \frametitle{Operationen}
  
\end{frame}

\begin{frame}
  \frametitle{Beispiele}

\end{frame}

\subsection{Code Signierung}

\begin{frame}
  \frametitle{Workflow}
  \begin{itemize}
  \item Entwickler signiert Code mit GPG
  \item Auditor signiert Code mit GPG
  \item zentrale Speicherung der Signaturen
  \item Verifikation des Codes durch Benutzer
  \item Feststellung der am wenigsten vertrauensw\"urdigen Codezeilen
  \end{itemize}
\end{frame}

\begin{frame}
  \frametitle{Trust model}
  \begin{itemize}
  \item Bisher wird dem Rechner, auf dem das Repository liegt, vertraut
  \item Einbr\"uche in diesen Rechner hat verheerende Folgen
  \item Vertrauen wird vom Repository zum Rechner der einzelnen Entwickler \"ubertragen
  \item Durch Web of trust kann bestimmten Leuten mehr vertraut werden als anderen Leuten
  \end{itemize}
\end{frame}

\begin{frame}
  \frametitle{Konzept}
  \begin{itemize}
  \item Signatur erfolgt zeilenbasiert pro Datei
  \item Alle verfuegbaren Signaturen werden verifiziert
  \item Jeder Change wird signiert
  \item Externe koennen Code reviewen und signieren
  \item M\"ogliche Authentifizierung anhand des GPG-Keys
  \end{itemize}
\end{frame}

\section{Implementierung}

\begin{frame}
  \frametitle{Orthoganale Entwicklung}
  \begin{columns}
    \begin{column}{5cm}
      
    \end{column}
    \begin{column}{7cm}
      \begin{block}{Orthogonaler Entwurf}
        \begin{itemize}
        \item
        \end{itemize}
      \end{block}

    \end{column}
  \end{columns}
\end{frame}

\begin{frame}
  \frametitle{Technik}
  \begin{itemize}
  \item \COMMONLISP{}
  \item \LIBGPGME{}
  \item \OSICAT{}
  \item \PORTPATH{}
  \end{itemize}
\end{frame}

\section{Zusammenfassung}

\begin{frame}
  \frametitle{Weiter Informationen}
  \begin{block}{Gennf}
    \begin{itemize}
    \item \tt{http://gennf.berlios.de}
    \item \tt{http://gennf.berlios.de/scratchpad.pdf}
    \end{itemize}
  \end{block}
  \begin{block}{Refernzen}
    \begin{itemize}
    \item \tt{http://www.gnupg.org/related\_software/gpgme/}
    \item \tt{http://http:users.footprints.net/\~kaz/mcvs.html}
    \item \tt{http://sbcl.sourceforge.net/}
    \end{itemize}
  \end{block}
\end{frame}

\end{document}
