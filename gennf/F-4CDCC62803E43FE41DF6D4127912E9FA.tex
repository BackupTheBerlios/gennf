\documentclass[german]{beamer}

\newcommand{\GPG}{\textsc{gpg}}
\newcommand{\PGP}{\textsc{pgp}}
\newcommand{\SBCL}{\textsc{sbcl}}
\newcommand{\GENNF}{\textsc{gennf}}
\newcommand{\METACVS}{\textsc{metacvs}}
\newcommand{\CVS}{\textsc{cvs}}
\newcommand{\DARCS}{\textsc{darcs}}
\newcommand{\GNUARCH}{\textsc{gnu\,arch}}
\newcommand{\MERCURIAL}{\textsc{mercurial}}
\newcommand{\SVK}{\textsc{svk}}
\newcommand{\LISP}{\textsc{lisp}}
\newcommand{\COMMONLISP}{\textsc{common lisp}}
\newcommand{\LIBGPGME}{\textsc{libgpgme}}
\newcommand{\OSICAT}{\textsc{osicat}}
\newcommand{\PORTPATH}{\textsc{portpath}}
\newcommand{\POSIX}{\textsc{posix}}
\newcommand{\HAKEN}{\ding{51}}

\usepackage{babel}
\usepackage{stmaryrd}
\usepackage{pifont}

\usetheme{Rochester}
\usefonttheme{serif}
\usecolortheme[RGB={120,0,16}]{structure}
\setbeamertemplate{navigation symbols}{}

\setbeamertemplate{footline}{
  \setbeamercolor{footer}{fg=white,bg=structure}
  \begin{beamercolorbox}{footer}
    \vskip2pt
    \hspace{1em}Infrastrukturen zur Open Source Softwareentwicklung\hfill
    \insertframenumber\,/\,\inserttotalframenumber\hspace{1em}
    \vskip2pt
  \end{beamercolorbox}
}

\author{Fabian~Otto~{\scriptsize (\texttt{sigsegv@cs.tu-berlin.de})}
  \and Hannes~Mehnert~{\scriptsize (\texttt{mehnert@cs.tu-berlin.de})}
  \and Florian~Lorenzen~{\scriptsize (\texttt{florenz@cs.tu-berlin.de})}}

\title{\GENNF}
\subtitle{Verteiltes Versionsmanagement mit Code Signierung}
\institute{TU~Berlin, FG Formale Modelle, Logik und Programmierung, \\
Infrastrukturen zur Open Source Softwareentwicklung WS05/06 \\
Bernd Mahr, Steffen Evers}
\date{31.~Januar~2006}

\begin{document}

\AtBeginSection[] {
  \begin{frame}
    \frametitle{"Ubersicht}
    \tableofcontents[currentsection]
  \end{frame}
}

\AtBeginSubsection[] {
  \begin{frame}
    \frametitle{"Ubersicht}
    \tableofcontents[currentsubsection]
  \end{frame}
}

\frame[plain]{\titlepage}

\begin{frame}
  \frametitle{"Ubersicht}
  \tableofcontents
\end{frame}


\section{Idee}

\begin{frame}
  \frametitle{Zielsetzung}
  \begin{block}{Verifikation der Herkunft des verwalteten Codes}
    \begin{itemize}
    \item Bisher mu"s dem Repository Server vertraut werden
    \item Code soll vertrauensw"urdig sein, durch Signaturen von Entwicklern
    \end{itemize}
  \end{block}
  \begin{block}{Dezentrale Repositories}
    \begin{itemize}
    \item Anpassung der Repository- an die Arbeitsstruktur des
      Projektes
    \item Keine vollst"andige Replikation von Repositories
      $\to$ Vernetzung durch Links
    \item Einfaches Verzweigen und Zusammenf"uhren von
      Entwicklungslinien
    \end{itemize}
  \end{block}
\end{frame}

% \begin{frame}
%   %% \frametitle{Umfeld/existierende L"osungen}
%   \frametitle{\CVS, \DARCS, \GNUARCH, \MERCURIAL, \SVK, \METACVS}
%   % \begin{itemize}
%   % \item
%   % \end{itemize}
%   \begin{block}{Vertrauen}
%     \begin{itemize}
%     \item Teilweise Signierung von Sourcecode
%     \item \alert{Keine} Buchpr"ufung von Sourcecode (Auditing)
%     \end{itemize}
%   \end{block}

%   \begin{block}{Verteilung}
%     \begin{itemize}
%     \item \alert{Keine} Benutzung vorhandener Infrastruktur\\
%       (SourceForge, BerliOS, \ldots)
%     \item Portabilit"at
%     \end{itemize}
%   \end{block}
  
%   % \begin{block}{Verteilung}
%   %   \begin{itemize}
%   %   \item Zumeist vollst"andige Replikation des Repositories
%   %   \item Keine Vernetzung?
%   %   \item vorhandene Infrastrukturen nicht immer genutzt.
%   %   \end{itemize}
%   % \end{block}
%   % \begin{block}{Signierung}
%   %   \begin{itemize}
%   %   \item Darcs (mail)
%   %   \item GNU Arch (Repositories)
%   %   \end{itemize}
%   % \end{block}
% \end{frame}

\begin{frame}
  \frametitle{\METACVS}
  \begin{block}{Vorteile}% von \METACVS}
    \begin{itemize}
    \item Gute Verf"ugbarkeit, da \CVS{} als Backend
    \item Versionierte Dateien in einer flachen Struktur (F-Dateien)
    \item Verzeichnisse werden durch separate Abbildung versioniert
    \item<2> \emph{Diese Ideen "ubernehmen wir}
    \end{itemize}
  \end{block}
  \begin{block}{Nachteile}<2>
    \begin{itemize}
    \item "Anderungen nur auf Dateiebene
    \item Starke Orientierung an \CVS{}
    \end{itemize}
  \end{block}
\end{frame}


\begin{frame}
  \frametitle{Planung}
  \begin{enumerate}
  \item Festlegung des Projektziels \only<2>{\HAKEN}
  \item Konzeptionierung:
    \begin{itemize}
    \item Verteilung \only<2>{\HAKEN}
    \item Code Signierung \only<2>{\HAKEN}
    \end{itemize}
  \item Implementierung eines Prototypens
    \begin{itemize}
    \item \CVS{}-Anbindung 
    \item Verteilung
    \item Anbindung zur Signierung
    \end{itemize}
  \end{enumerate}
\end{frame}

\section{Konzept}

\begin{frame}
  \frametitle{Entwurf}
  \begin{block}{Verteilung}
    \begin{itemize}
    \item Abstraktes Modell f"ur verteilte Repositories
    \item Entwurf der Datenstrukturen und Operationen
    \item Typische Anwendungsf"alle
    \end{itemize}
  \end{block}
  \begin{block}{Code Signierung}
    \begin{itemize}
    \item Workflow
    \item Trust Model
    \item Konzept
    \end{itemize}
  \end{block}
\end{frame}

\subsection{Verteiltes Versionsmanagement}
\newcommand{\STRUCT}[1]{\ensuremath{\mathnormal{#1}}}
\newcommand{\NORMAL}[1]{\ensuremath{\mbox{\textnormal{#1}}}}
\def\APPEND{:\mathrel\cdot}

\begin{frame}
  \frametitle{Strukturen}

  \begin{block}{Definition: Meta-Datenstrukturen}
    \medskip
    $\begin{array}{@{\hspace{1em}}lccl}
      \STRUCT{repository} & r & \equiv & \{b\ |\ b : \STRUCT{branch}\} \\
      \STRUCT{branch}     & b & \equiv &
      (\beta, s),\ s : \STRUCT{change\,sequence} \\
      \STRUCT{change}     & C & \equiv &
      \STRUCT{commit}\oplus\STRUCT{merge} \\
      \STRUCT{commit}     & c & \equiv &
      (\nu, \beta, \alpha, f),\ f : \STRUCT{filename}\to
      \STRUCT{revision} \\
      \STRUCT{merge}      & m & \equiv &
      (\nu, \beta, \alpha, f, o),\ o : \STRUCT{origin} \\
      \STRUCT{origin}     & o & \equiv &
      (\nu, \beta, \alpha)      \end{array}$
  \end{block}

  \begin{block}{Beispiel: Neu angelegtes Projekt}
    \medskip
    \begin{tabular}{@{\hspace{1em}}ll}
      $\begin{array}{@{}lcl}
        r   &=& \{b_1\} \\
        b_1 &=& (1, \langle c_1 \rangle) \\
        c_1 &=& (1, 1, \mathsf{cvs}, f_1)
      \end{array}$ &
      $\begin{array}{lcl}
        f_1 &=& \left[
          \begin{array}{lcl}
            \mathtt{F-}a & \mapsto & \mathsf{r1.1} \\
            \mathtt{F-}b & \mapsto & \mathsf{r1.1} \\
            \mathtt{F-}c & \mapsto & \mathsf{r1.1} \\
            \mathtt{MAP} & \mapsto & \mathsf{r1.1}
          \end{array}
        \right]
      \end{array}$
    \end{tabular}
  \end{block}
\end{frame}

\begin{frame}
  \frametitle{Operationen}

  \begin{block}{Definition: Branching und Merging}
    \medskip
    \hspace{1em}Seien $b, b'$ Branches und $r=\{b, b'\}$ ein Repository mit

    \hspace{1em}$b=(\beta, \langle C_1, \ldots, C_k\rangle)$ und $b'$ analog.
    \medskip

    $\begin{array}{@{\hspace{1em}}lrcl}
    \STRUCT{merging}      & C_i \rhd C_k'              & \equiv &
    (\beta', s') \leadsto
    (\beta', s'\APPEND m)\ \NORMAL{mit} \\
                          &                            &        &
                          \begin{array}{@{}lcl}
                          m   &=& (\nu, \beta', \alpha, f, o), \\
                          o   &=& (\nu_i, \beta_i, \alpha_i) \\
                          \end{array} \\
    \STRUCT{branching}    & \triangledown C_i          & \equiv &
    C_i\rhd \boxempty \quad \NORMAL{mit $\boxempty$ leerer Branch}
%
% Dies war zuviel (ausserdem sieht es genauso aus wie bei merging).
%     R \Leadsto R\ \Cup \{B_2\}\ \Normal{Mit} \\
%                           &                            &        &
%                           \Begin{Array}{@{}Lcl}
%                           B_2 &=& (\Beta_2, \Langle M\Rangle), \\
%                           M   &=& (\Nu, \Beta_2, \Alpha, F, O), \\
%                           O   &=& (\Nu_I, \Beta_I, \Alpha_I) \\
%                           \end{Array}
%
    \end{array}$
  \end{block}

  \begin{block}{Anmerkungen}
    \begin{itemize}
    \item Interaktion Repository $\leftrightarrow$ Sandbox:
      $\STRUCT{commit}$, $\STRUCT{update}$
    \item Bei einem Merge kann es Konflikte geben
    \item Bei einem Branch kann Replikation stattfinden:
      $\underline\triangledown_nC_i$
    \end{itemize}
  \end{block}
\end{frame}

\begin{frame}
  \frametitle{Beispiele}
  \begin{columns}
    \begin{column}{0.5\textwidth}
      \begin{flushleft}
        \only<1>{\input{beispiel0.pdf_t}}
        \only<2>{\input{beispiel1.pdf_t}}
      \end{flushleft}
    \end{column}
    \begin{column}{0.5\textwidth}
      \begin{block}<1->{Stern-Topologie}
        \begin{itemize}
        \item Ein Haupt-Repository f\"ur "`gereiften"' Code
        \item Lokale Repositories bei jedem Entwickler/Team
        \end{itemize}
      \end{block}
      \begin{block}<2>{Baum-Topologie}
        \begin{itemize}
        \item Weitere Hierarchien f"ur Teilprojekte
        \item Code wandert nach oben
        \end{itemize}
      \end{block}
    \end{column}
  \end{columns}
\end{frame}

\subsection{Code Signierung}

\begin{frame}
  \frametitle{Workflow}
  \begin{block}{Entwickler}
    \begin{itemize}
      \item Signiert Change mit \GPG
    \end{itemize}
  \end{block}
  \begin{block}{Benutzer}
    \begin{itemize}
      \item Verifikation der Signaturen des Codes
      \item Erkennung der wenig vertrauensw"urdigen Codestellen
    \end{itemize}
  \end{block}
  \begin{block}{Auditor}
    \begin{itemize}
      \item Vertraut gelesenem Code, signiert ihn mit \GPG{}
      \item Benutzer kann somit zum Auditor werden
    \end{itemize}
  \end{block}
\end{frame}

\begin{frame}
  \frametitle{Trust Model}
  \begin{block}{Bisher}
    \begin{itemize}
      \item Repository-Rechner mu"s vertraut werden
      \item Ein m"oglicher Einbruch kann beliebig Code "andern
      \item Einbruch wird m"oglicherweise erst sp"at oder gar nicht bemerkt
    \end{itemize}
  \end{block}
  \begin{block}{Neues Trust Model}
    \begin{itemize}
      \item Vertrauen wird vom Repository Rechner zu denen
         der einzelnen Signierer "ubertragen
      \item Privater Schl"ussel eines Signierers kann widerrufen werden
      \item Web of Trust, jeder Benutzer kann jedem Signierer
         einen (pers"onlichen) Trust Value zuteilen
    \end{itemize}
  \end{block}
\end{frame}

\begin{frame}
  \frametitle{Code Signierung - Implementationskonzept}
  \begin{block}{Erste Schritte}
    \begin{itemize}
      \item Changes
      \item \GPG{}-Wrapper f"ur \SBCL
      \item Meta-Datei, in der die Signaturen gespeichert werden
    \end{itemize}
  \end{block}
  \begin{block}{Features}
    \begin{itemize}
      \item Authentifizierung "uber \GPG{}, damit ein Auditor keinen Zugriff
        auf das Repository braucht (um zu signieren)
      \item Automatische Verifikation der Changes, auf dem Server
      \item Automatisierte Suchen von wenig signierten und wenig
        vertrauensvoll signierten Codest"ucken
      \item Revocation von Signaturen
    \end{itemize}
  \end{block}
\end{frame}

\section{Implementierung}

\begin{frame}
  \frametitle{Orthoganale Entwicklung}
  \begin{columns}[c]
    \begin{column}{0.5\textwidth}
      \input{orthogonal.pdf_t}
    \end{column}
    \begin{column}{0.5\textwidth}
      \begin{block}{Orthogonaler Entwurf}
        \begin{itemize}
        \item Verteilung 
        \item Signierung
        \item Struktur
        \end{itemize}
      \end{block}
      \begin{block}{Changes}
        \begin{itemize}
        \item Verteilung und Signierung ben"otigen
          Changes
        \item Haben deswegen erste Priorit"at
        \end{itemize}
      \end{block}

    \end{column}
  \end{columns}
\end{frame}

\begin{frame}
  \frametitle{Technik}
  \begin{block}{Sprache und Bibliotheken}
    \begin{itemize}
    \item \SBCL{} (Steel Bank Common Lisp)
    \item \OSICAT{} (\POSIX{} Bibliothek)
    \item \PORTPATH{} (Pfadnamen Bibliothek)
    \item \LIBGPGME{} (\PGP{} Bibliothek)
    \end{itemize}
  \end{block}
  \begin{block}{Angebundene Programme}
    \begin{itemize}
    \item \CVS{} (Concurrent Version System)
    \end{itemize}
  \end{block}
\end{frame}

%%\section{Zusammenfassung}

\begin{frame}
  \frametitle{Weiterf"uhrende Informationen}
  \begin{block}{\GENNF}
    \begin{itemize}
    \item \tt{http://gennf.berlios.de/}
    \item \tt{http://gennf.berlios.de/scratchpad.pdf}
    \end{itemize}
  \end{block}
  \begin{block}{Referenzen}
    \begin{itemize}
    \item \tt{http://www.gnupg.org/related\_software/gpgme/}
    \item \tt{http://users.footprints.net/\textasciitilde{}kaz/mcvs.html}
    \item \tt{http://sbcl.sourceforge.net/}
    \item \tt{http://www.cvshome.org/}
    \end{itemize}
  \end{block}
\end{frame}

\end{document}
