\documentclass[german]{beamer}

\newcommand{\GENNF}{\textsc{gennf}}
\newcommand{\METACVS}{\textsc{metacvs}}
\newcommand{\CVS}{\textsc{cvs}}
\newcommand{\LISP}{\textsc{lisp}}
\newcommand{\COMMONLISP}{\textsc{common lisp}}
\newcommand{\LIBGPGME}{\textsc{libgpgme}}
\newcommand{\OSICAT}{\textsc{osicat}}
\newcommand{\PORTPATH}{\textsc{portpath}}

\usepackage{babel}
\usetheme{Rochester}
\usefonttheme{serif}
\usecolortheme[RGB={120,0,16}]{structure}

\setbeamertemplate{navigation symbols}{}

\author{Fabian~Otto~{\scriptsize (\texttt{sigsegv@cs.tu-berlin.de})}
  \and Hannes~Mehnert~{\scriptsize (\texttt{mehnert@cs.tu-berlin.de})}
  \and Florian~Lorenzen~{\scriptsize (\texttt{florenz@cs.tu-berlin.de})}}

\title{\GENNF}
\subtitle{Verteiltes Versionsmanagement mit Code Signierung}
\institute{TU~Berlin, FG Formale Modelle, Logik und Programmierung, \\
Infrastrukturen zur Open Source Softwareentwicklung WS05/06 \\
Bernd Mahr, Steffen Evers}
\date{31.~Januar~2006}


\begin{document}
\AtBeginSection[]
{
  \begin{frame}
    \frametitle{"Ubersicht}
    \tableofcontents[currentsection]
  \end{frame}
}

\AtBeginSubsection[]
{
  \begin{frame}
    \frametitle{"Ubersicht}
    \tableofcontents[currentsubsection]
  \end{frame}
}

\frame[plain]{\titlepage}

\begin{frame}
  \frametitle{"Ubersicht}
  \tableofcontents
\end{frame}

\section{Idee}

\begin{frame}
  \frametitle{Zielsetzung}
  \begin{itemize}
  \item Sicherstellung der Herkunft des verwalteten Codes
  \item Kein zentrales Repository:
    \begin{itemize}
    \item Anpassung der Repository- an die Arbeitsstruktur des Projektes
    \item Keine vollst"andige Replikation von Repositories $\rightarrow$
      Vernetzung durch Links
    \item Einfaches Verzweigen und Zusammenf"uhren von Entwicklungslinien
    \end{itemize}
  \item Gro"se Verf"ugbarkeit
  \end{itemize}
\end{frame}

\begin{frame}
  \frametitle{Umfeld}
  \begin{itemize}
  \item Zumeist vollst"andige Replikation des Repositories
  \item Keine Vernetzung
  \end{itemize}
  vorhandene verteilte Software:
  \begin{itemize}
  \item darcs
  \item tal (GNU Arch)
  \end{itemize}
\end{frame}

\begin{frame}
  \frametitle{\METACVS}
  L"ost Problem der Verzeichnisversionierung in \CVS:
  \begin{itemize}
  \item Abspeicherung einer flachen Struktur
  \item Verzeichnishierarchie wird "uber separate Abblidung
    gespeichert
  \item Clientseitige Erweiterung, ben"otigt \CVS{} als Backend
    $\rightarrow$ gute Verf"ugbarkeit
  \item \emph{Diese Ideen "ubernehmen wir}
  \end{itemize}
\end{frame}

\begin{frame}
  \frametitle{Planung}
  \begin{enumerate}
  \item Festlegung des Projektziels
  \item Konzeptionierung:
    \begin{itemize}
    \item Verteilung
    \item Code Signierung
    \end{itemize}
  \item Implementierung eines Prototypens
  \end{enumerate}
\end{frame}

\section{Konzept}

\begin{frame}
  \frametitle{Entwurf}
  \begin{itemize}
  \item Verteilung
    \begin{itemize}
    \item Abstraktes Modell f"ur verteilte Repositories
    \item Entwurf der Datenstrukturen und Operationen
    \item Implementierungshinweise
    \item Typische Anwendungsf"alle
    \end{itemize}
  \item Code Signierung
  \end{itemize}
\end{frame}

\subsection{Verteiltes Versionsmanagement}

\begin{frame}
  \frametitle{Strukturen}
  
\end{frame}

\begin{frame}
  \frametitle{Operationen}
  
\end{frame}

\begin{frame}
  \frametitle{Beispiele}

\end{frame}

\subsection{Code Signierung}

\begin{frame}
  \frametitle{Workflow}
  
\end{frame}

\section{Implementierung}

\begin{frame}
  \frametitle{Orthoganale Entwicklung}
  
\end{frame}

\begin{frame}
  \frametitle{Technik}
  \begin{itemize}
  \item \COMMONLISP{}
  \item \LIBGPGME{}
  \item \OSICAT{}
  \item \PORTPATH{}
  \end{itemize}
\end{frame}

\section{Zusammenfassung}

\begin{frame}
  \frametitle{Zusammenfassung}
  
\end{frame}
\end{document}
