% Copyright 2006 Hannes Mehnert, Florian Lorenzen, Fabian Otto
%
% This file is part of gennf.
%
% gennf is free software; you can redistribute it and/or modify it
% under the terms of the GNU General Public License as published by
% the Free Software Foundation; either version 2 of the License, or
% (at your option) any later version.
%
% gennf is distributed in the hope that it will be useful, but WITHOUT
% ANY WARRANTY; without even the implied warranty of MERCHANTABILITY
% or FITNESS FOR A PARTICULAR PURPOSE.  See the GNU General Public
% License for more details.
%
% You should have received a copy of the GNU General Public License
% along with gennf; if not, write to the Free Software Foundation,
% Inc., 51 Franklin St, Fifth Floor, Boston, MA 02110-1301 USA
%
% $Id: F-881560526E7214793C24206DE07FE66D.lisp,v 1.8 2006/02/18
% 10:45:02 sigsegv Exp $

% This is the project's report stateing on the course of development
% as well as on design and technical issues.

\documentclass[fleqn, 10pt, a4paper]{report} \usepackage{amssymb}
\usepackage{stmaryrd} \setlength{\parindent}{0pt}

\newcommand{\GENNF}{\textsc{gennf}}

\begin{document}

\thispagestyle{empty}
\begin{center}
  {\scshape Infrastrukturen zur Open Source Softwareentwicklung} \\
  WS05/06, TU~Berlin, FG Formal Modelle, Logik und
  Programmierung \\
  Prof.~Dr.~Bernd~Mahr, Steffen~Evers \bigskip \bigskip

  {\LARGE \GENNF{} -- Distributed Version Management System with Code
    Signing} \medskip

  {\Large -- Project Report --} \bigskip \bigskip

  {\large
    \begin{minipage}[t]{0.35\linewidth}
      \begin{center}
        Fabian~Otto \\ \vspace{-2mm} {\scriptsize\ttfamily
          sigsegv@cs.tu-berlin.de}
      \end{center}
    \end{minipage}\hspace{8mm}
    \begin{minipage}[t]{0.35\linewidth}
      \begin{center}
        Florian~Lorenzen \\ \vspace{-2mm} {\scriptsize\ttfamily
          florenz@cs.tu-berlin.de}
      \end{center}
    \end{minipage}
    \bigskip \bigskip

    \today}

\end{center}

\tableofcontents
\newpage

\section*{Prologue}
\addcontentsline{toc}{section}{Prologue}

This document is the project report of \GENNF{}, a distributed version
management system enhanced with code signing as part of the
\emph{Infrastrukturen zur Open Source Software Entwicklung} course
held by Steffen Evers at Berlin Technical University.

The document started as a scratchpad we sketched our ideas and plans
and is currently being transformed into an intermediate report and
once will be the final report.

The documente is divided into two chapters:
\begin{itemize}
\item The first chapter describes the course of the project in
  chronologically order. Decisions and steps we undertook are
  explained as well as comments from a review perspective.
\item The second chapter describes the desgin and concepts of \GENNF{}
  in a more technical fashion. Large parts of this section were
  written as a desgin document in an early period of the project, some
  parts were corrected and adjusted during development or thereafter.
  Thus, this chapter can be considered as a part of the project's
  outcome.
\end{itemize}

\section*{Developers}
\addcontentsline{toc}{section}{Developers}

\GENNF{} is being developed by
\begin{itemize}
\item Fabian Otto, \texttt{sigsegv@cs.tu-berlin.de}
\item Florian Lorenzen, \texttt{florenz@cs.tu-berlin.de}
\end{itemize}


\chapter{The project's course \label{course}}

This chapter describes in rough chronological order the different
steps we undertook during the development of \GENNF{}.

\section{The planning}

The planning phase covers all the time prior to any coding. As we
changed plans at several points there is some interleaving during the
middle period of the project.

\subsection{In search of a project}

In the time from mid of October to begin of November we were looking
for this term's project. At end of October it was clear that at least
Hannes Mehnert and Florian Lorenzen would risk a try in the area of
version management software. Fabian Otto definitely joined in during
the first days of November.

During the first half of November lots of issues concerning already
existent software and the project's goals were discussed.

Among them were a few we abandoned:

\subsubsection{Plans we abandoned}

We thought about improving the code merging capabilities which are
heavily relied on in version control software. The idea was to do a
merge not of streams of characters (the source code) but of abstract
syntax trees of the programming language used. This seemd very
promising because fewer conflicts should arise. We abandoned this plan
for several reasons:

\begin{itemize}
\item It is language dependant: for every language used in the code at
  least a context-free and a regular grammar are required. To do the
  lexing and parsing a lexer- and parser-generator or some function
  $\phi : G_L \rightarrow G_P \rightarrow S \rightarrow T$ with $G_L$
  being a lexer grammar, $G_P$ a parser grammar, $S$ a source code and
  $T$ an abstract syntax tree would have to be implemented. This is
  not something impossible but takes quite some time.
\item The algorithm to compare two abstract syntax trees is not so
  trivial, at least if something better than line based diffs is
  supposed to come out. Another trouble is what to consider as a
  conflict.
\item How to store diffs of to trees? Our idea was to store functions
  $\tau : T \rightarrow T$ which transform one tree into the other.
  Those function could also be used to apply the changes. At this
  point the question arose if $\tau_1 \circ \tau_2 \neq \tau_2 \circ
  \tau_1$ could be used as the definition of a conflict. But this is
  probably to simple.  Those function $\tau_i$ in turn would be
  compositions of atomic transformations of one node in the tree,
  i.~e. a combinataion of paths in the tree and actions to be taken at
  the end of the path.
\item Preserving the code layout may be a problem, even with layout
  annotations in the syntax trees.
\item What to use as the storage backend? Some line based store would
  not really fulfil the needs because this meant to pretty-print the
  syntax tree (using the layout annotations) and to parse the last
  checked in version at each commit. Consequently, the transformation
  function should be stored. This in the end lead to the decision to
  forget about improving the merge because the best way to do it is to
  write something completely new which is too much work for one
  semester.
\end{itemize}

As we found out later in February 2006, exactly this approach was
taken by Hayase et. al. \cite{HMI05}.

\subsubsection{Plans that should not stay only on paper}

We more or less settled on two other aspects which seemed improvable
to us:

\begin{itemize}
\item Code signing
\item Ditributed repositories
\end{itemize}

Code signing means, that each chunk of code checked into the
repository has to be signed by the programmer using some
public/private key scheme to enable other developers or users to
verify the origin of the code (as long as keys are trustworthy).

Distributed means that the repository is not stored on a single server
but that commits can be done on several servers and the results be
merged. The advantages are a better failure behaviour and the
possibility to structure the repositories according to some external
requirements, e.~g. responsibilities for or maintenance of certain
code sections by certain parties.

The code signing part should be mainly worked on by Hannes Mehnert,
whereas Florian Lorenzen felt more reponsible for the distributive
part. Fabian Otto would participate in both parts of the project.

It was fairly clear right from the beginning that we could not and
want not write a version management system from scratch but to
improcve an existing system. The two systems in question were
\textsc{darcs} and \textsc{metacvs}.

\textsc{darcs} is already a distributed system -- but it is too
distributed for our taste: every working copy is a repository.  This
should not be necessary and one option was to offer greater
flexibility concering repositores and working copies in
\textsc{darcs}.

\textsc{metacvs} is a wrapper around \textsc{cvs} and requires only
\textsc{cvs} onb the server side. It circumvents some quirks of
\textsc{cvs}, especially the weak handling of directory structure of a
project. Like \textsc{cvs} it is a central-server oriented system and
consequently the option as to introduce distribution into this system.

Concerning code signing neither of the two had support in our sense
(\textsc{darcs} has some rudimentary support using digitally signed
mails).

We decided to build on \textsc{metacvs} because it is, in turn, based
on \textsc{cvs} which we know quite well and because it is written in
\textsc{Common~Lisp} which makes rapid development at least possible.
Another reason is that \textsc{cvs} is widely available on services
like SourceForge or BerliOS and thus existing infrastructure is
available to the new system.

We also had a look on a system called \textsc{dcvs}, distributed
\textsc{cvs} but this was ruled out right from the start, beacuse its
support several repositories is rather unflexible: each repository is
assigned a part of code which is the only one that can be modified.
That means the distribution must follow the responsibilities and/or
maintenanceships of the projects which imposes to strong rules on the
development process in our opinion.

\subsection{Infrastructure}

Around the same time that we had stated the two overall goals of the
project we applied for a project at BerliOS which was approved on
November 7 with the following proposal:

\begin{verbatim}
As part of the course "Infrastructures for Open Source Software
Development" [1] held by Steffen Evers at Berlin Technical University
[2], we want to implement a distributed version management system with
code signing support.

Code signing intends to open the possibility that the origin of each
chunk of code of a piece of software can be verified using
public/private key signing. Without code signing users and other
developers have to trust the integrity of the machine hosting the
sources.

Distributed version management means that there is no need for a central
repository which avoids single points of failure and provides means to
structure the repositories according to some external criteria like
responsibilities or maintenanceships. We want to implement a system
without the necessity that each site hosting a repository has to
store all revisions of projects which can cause huge storage
consumption.

We want to implement the system as an extension of Meta-CVS [3], which
is open source under the GPL license. The code signing will 
heavily rely on GnuPG [4].
Meta-CVS is written in CLISP [5], which is an open source implementation
of Common Lisp [6]. To ease development, our code will be written in
Common Lisp (using CLISP extension when necessary).

The name of the project will be gennf.

----------
References:
[1] http://flp.cs.tu-berlin.de/lehre/wise05-06/ossi.html
[2] http://www.tu-berlin.de
[3] http://users.footprints.net/~kaz/mcvs.html
[4] http://www.gnupg.org/
[5] http://clisp.cons.org/
[6] http://www.cs.cmu.edu/Groups/AI/html/cltl/cltl2.html}
\end{verbatim}

We installed a \textsc{cvs} repository at
\texttt{cvs.gennf.berlios.de:/cvsroot/gennf} which we used with
\textsc{metacvs}. For this reason the \textsc{viewcvs} interface of
BerliOS is not so usable, because it cannot resolve \textsc{metacvs}'s
F-files.

The first version of our web-page at \texttt{http://gennf.berlios.de}
was online about a month later on December 12.

To ease communication among the developers we installed an publicly
open mailing list \texttt{gennf-devel@lists.berlios.de} on
November~30. Some days later commit mails were enabled.  The Perl
script \texttt{log\_accum.pl} used originally stems from the FreeBSD
project and found the way to us via the ZEDV group at Berlin's Free
University's physics department and the \texttt{octopus} project.
Until mid of January it had the same problem as \textsc{viewcvs}: it
could not resolve \textsc{metacvs}'s F-file names, until a very basic
\texttt{MAP}-file parser was included into the script.

The mailing list archives are found at
\texttt{http://lists.berlios.de/pipermail/gennf-devel/} and as all
commit mails were sent to and many discussions took place on the list
it reads like a (not properly edited, though) logbook of the project.

\subsection{Designing and planning the implementation}

Until mid of november we had did not have a schedule or distinct
milestones. The reason for this was twofold: Firstly, it took some
time to get the goals of the project straight. Secondly, we had the
problem that among the three developers was one (Florian Lorenzen) who
had to do most of the work until the mid of February 2006 and two
others (hence Hannes Mehnert and Fabian Otto) who would not start
intensive work before that point of time. That constrained us to plan
a handing over of the code at around the middle of February in a
defined state.

Having this discussed it was made up the plan, that the basic
operations for the handling of distributed repositories should be
protypically implemented until then. The directory structure, the code
signing and the user-interface should be built on top of this during the
following one and a half months.

A concept how the distribution shall take place had to be finished
until end of the year 2005. This was done in several steps and finally
completed on January 3, 2005.

A more in depth description of the requirements of the code signing
was written mid of November.

Our schedule then looked like this: \bigskip

\begin{tabular}{|p{.3\textwidth}|p{.6\textwidth}|}
  \hline
  \textmd{When} & \textmd{What} \\
  \hline\hline
  Nov. 15, 2005 to Jan. 1, 2006 &
  Getting familiar with the code of \textsc{metacvs}
  and drafting a design for distributiveness and signing. \\
  \hline
  Dec. 15, 2005 to Jan. 1 2006 &
  Porting \textsc{metacvs} to \textsc{sbcl}. \\
  \hline
  Jan. 9 to Feb. 19, 2006 & Prototypic implementation of
  distributed repositories (by Florian Lorenzen). \\
  \hline
  Jan. 24 to Jan. 30, 2006 & Preparation of the intermediate
  presentation. \\
  \hline
  Feb. 20 to Mar. 31, 2006 & Implementation of the code signing
  and the directory structure (by Hannes Mehnert and Fabian Otto). \\
  \hline
  Apr. 1 to Apr. 6, 2006 & Finishing the project's report
  and preparing the presentation for final colloquium. \\
  \hline
  Apr. 11, 2006, 14:00 & Final colloquium, 20 minutes talk. \\
  \hline
\end{tabular}

\subsection{Throw over of the plan}

On March 10, the team receives a mail message from Hannes Mehnert that he
quits the project due to more important things to do. As he has not written
a single line of code (apart from some \TeX-code in section \ref{codesigning})
there is nothing for someone else has to take over.

But the reduced number of team members causes cutting features, of course.

\subsection{Working on \textsc{metacvs}}

As we decided to extend \textsc{metacvs} we started investigating the
code of version 1.0.13 in December.

The system is written in \textsc{clisp}, a particular
\textsc{Common~Lisp} implementation. For the code signing part we
wanted to rely on the \textsc{gpgme} library. For this reason
including \textsc{c}-libraries was an important issue. At some point
in the mid of December we decided to use \textsc{sbcl} as the
\textsc{lisp} implementation of choice because it supports
\textsc{uffi}, which is a cross-platform foreign function interface,
and is less clumsy than \textsc{clisp} at some points.

This implied porting \textsc{metacvs} to \textsc{sbcl}. After close
looks on the implementation dependent parts of the code it was clear
that most of them could be replaced by a portable library called
\textsc{osicat}, which is a high level \textsc{posix} interface.
Inclusion of this library would make the code cleaner and more
portable. Furthermore, we decided that the current build system of
\textsc{metacvs} is very clumsy and that it should be replaced by the
widely used \textsc{asdf}-package to compile and load
\textsc{Common~Lisp} software.

The system was ported to \textsc{sbcl} at the end of December 2005.
This turned out to be a waste of time some days later:

After the specification for the distributed repositories was finished,
it turned out, that \textsc{metacvs} is to tightly written around
\textsc{cvs} to easily incorporate the modifications required for the
concept of changes. For the distributed repositories and the its
central concept of changes a clean backend abstraction, i.~e. a
\textsc{cvs} abstraction, was required. The reason is that quite a
different development model than \textsc{cvs}'s model should be
realised. \textsc{metacvs} did not change the development model in any
way but just improved \textsc{cvs} at some points.  The result of this
philosophy ws that basically a wrapper for each \textsc{cvs} operation
was provided and that the executable was called when needed.

Unfortunately, \textsc{metacvs} has no inner interface that we could
build on to benefit from its improvements over \textsc{cvs} and at the
same time abstracting from the sandbox centered development model. The
only possibility would have been wrapping the program once again which
would not be a very clean solution.

So we sorted out the aspect of \textsc{metacvs} that were very good
ideas, which are basically the indirection layer for directory
structure, and decided to implement a protype from scratch
incorporating these ideas (and code fragments from \textsc{metacvs} as
far as possible). At this point it was clear that a patch for
\textsc{metacvs} as project result was out of sight. It was
furthermore obvious that all the porting was not necessary at all.

From a review perspective it can be stated that the decision to take
\textsc{scbl} as development platform was not a good decision. If we
had sticked with \textsc{clisp} there would have been some quirks to
ship around but it would have been clear that patching was not
possible at least two weeks earlier. Unlucky enough those two weeks
were the academic holidays -- and there would have been plenty of time
to do useful codeing, instead of writing a superflouos port.

At beginning of January 2006 the codeing of a prototype to implement
the distribution model began.

\section{Implementation}

The implementation phase started at begin of January 2005.

\subsection{The distribution prototype}

The protoype for distribute repositories should provide all necessary
basic operations as described in section \ref{distrib} but only for
flat repositories, i.~e. without any directory structure. The idea
behind this was that all structure would be set on top of this as it
is done in \textsc{metacvs}.

The first task for this part was to find a suitable backend
abstraction. This turned out to be much more difficult than expected,
again because \textsc{cvs} imposes very strict rules on how the
development takes place. All those difficulties could somehow be
circumvented but it only with heavy traffic between the repository and
the client and massive use of temporary harddisk storage.

It now became clear why there is, apart from \textsc{dcvs}, no other
distributed version management system which builds on top of
\textsc{cvs}: \textsc{cvs} is to powerful for just using it as a store
and retrieve engine and network transport. Just using it for this
purpose needs a lot of tricks. Despite this insight \textsc{cvs} was
not replaced on the one hand side due to time constraints and on the
other hand side because it is widely available and indeed handles the
network for us.

But a more satisfing solution would have been to take an
transportation protocol like HTTP or FTP and some simpler software to
store the revisions of files.

Storeing revisions of files introduces the next problem we had because
of our sticking to \textsc{cvs}. Our concept of distribution heavily
relies on merging different revisions of files from different
branches. It turned out that \textsc{cvs} mergeing capabilities could
not be used for this because they only handle merges of files that
live in the same repository. Consequently, this merging had to be done
by ourselves. We use \textsc{cl-difflib} to compute differences
between files and some self-written routines to use this information
to merge two files. This is not a bad solution but it costed time to
implement of course. It also introduces the necessity to do the
merging ourselves even in situations in which we could have used
\textsc{cvs} because it is very undesireable to have two different
merging algorithms -- one we can control and one we cannot -- in one
piece of software.

Again, it would have been better to work without \textsc{cvs} and
using simpler software to store the revisions of files like
\textsc{Common~Lisp} versions of \texttt{diff} and \texttt{patch}.

However, it seems that the prototype could be finished until mid of
February -- though probably very buggy -- and can be handed to Fabian
Otto, to be enriched by 


\chapter{Concepts and desgin}

\section{Code Signing \label{codesigning}}

As Hannes Mehnert left the team Code Signing is not implemented and
this section is to be read as a draft what could have been done.

\subsection{Motivation}

Currently most version management systems don't use any digital
signatures for code (apart from \textsc{tla}). Authentication is done
by an own password server (\textsc{cvs}: pserver), via a ssh public
key (\textsc{cvs}, \textsc{subversion}, \textsc{darcs},
\textsc{metacvs}) or via a digitally signed mail (\textsc{darcs}).

On the other hand, data integrity is not supported.  Every user has to
trust the repository server to be secure from a possible hacker
breaking into it.  A hacker with access to the repository is able to
compromise the software by unperceived insert of a backdoor.

In the past, several big repositories have been hacked, including
KDE\footnote{
  \texttt{http://seclists.org/lists/fulldisclosure/2004/May/0349.html}},
SourceForge, \texttt{jabber.org}.


\subsection{Vision}

A developer using \textsc{gennf} should be able to sign his patches
by an already known-to-work and widely deployed public key
cryptosystem.

An auditor should be able to look at any number of parts of the code
and sign them digitally after a successful code audit.

A user checking out the code should be able to verify the code,
looking at the different trust levels of the lines of code. He should
be able to easily spot the lines which are least trusted according to
his web of trust. If he cares much about security, he can audit those
hot spots and sign them in the repository, becoming an auditor.

Thus, the trust is moved from the single point of failure (the
repository server) to every developer (able to protect his private
key, or otherwise revoke his key).

\subsection{Workflows}

A possible workflow for the developer would be that he first starts
coding, fixing a bug or implementing a new feature. When he is done
with that, during the commit he'll be asked whether he wants to sign
this commit.  If he chooses to sign, he is asked for the passphrase
protecting his private key. After successfully verification of the
passphrase, the commit will be signed and transferred to the server.
Authentication could probably also be done via the signature on the
code, so no user management is anymore needed on the server, only a
public keyring including all developer keys, but possibly anyone can
commit, and the user has to choose who to trust during checkout.

A user with access to the repository checking out can verify the
checked out code, and see how safe it is.  Via \texttt{gennf blame}
he should be able to look which lines are at which trust levels.

Via \texttt{gennf sign} a code auditor (or user) should be able to
sign parts of code (line-, file-based).

\subsection{Implementation overview}

\begin{itemize}
\item Public key cryptography interface for \textsc{sbcl}.
\item Unique patchset number, like in subversion, to match a signature
  to a revision.
\item Virtual patchsets describing regions in source files for
  auditor.
\item Decide where to store signatures.  One possibility is to provide a
  new meta file containg the signatures, what lines and revisions they
  apply to. Putting it this way, audits can be handled in the same
  mannner. Change and audit signatures have to distinguished by some
  tag. Storeing all this in a meta file makes audit signs an ordinary
  check in operation on this file -- which should be wrapped in a nice
  command of course. The same applies to normal commits: they change
  the F-files and make a change to the new meta file.
\item Integration in \textsc{gennf} commands like \texttt{commit},
  \texttt{checkout}, \texttt{blame}.
\item Implement more \textsc{gennf} commands like \texttt{verify},
  \texttt{sign}.
\end{itemize}

\subsection{Implementation detail}

OpenPGP will be used as pulic key cryptosystem, because it has an
open-source implementation \textsc{gpg} (GNU Privacy Guard), is not
hierarchically structured due to the web of trust and keyserver rings.
It is also widely deployed in open source developer communities.

It has a \texttt{GnuPG Made Easy} \textsc{gpgme} development library,
which needs to be wrapped in a C-Interface for \textsc{sbcl}.

The signatures has to be stored somewhere in \textsc{gennf},
probably in the meta files (cf. previous section).

Some parts should be configurable for \textsc{gennf} code signing, a
default key ID for the developer, a public keyring file for
authentication on the repository server, and possibly other things.

Probably an \textsc{emacs}-integration for \textsc{gennf} including
the audit-part would be useful.


\section{Distributed repositories \label{distrib}}

\subsection{Outline}

By support for \emph{distributed repository} we mean support for the
following features:

\begin{itemize}
\item \textsc{gennf} will provide a possibility to \emph{inherit} from
  an already existing repository which means to create a branch at
  another site. It is not necessary for this new site to store the
  whole history of the project: \textsc{gennf} will establish a link
  between the two.  Whenever someone at the new repository requests a
  version not locally available, it will be fetched from the original
  site. It is of course possible to replicate large parts of or the
  whole history to improve performance or redundancy. However,
  replicated parts of the history still know where they come from.

\item The different lines of development can be merged together.
  \textsc{gennf} keeps track of the origin of the code merged in. The
  mechanism of merge is used to create branches. In fact, a merge is
  the only basic operation \textsc{gennf} knows about, apart from
  ordinary commits (which do not come from another site but out of
  some developer's head).  Both -- commits and merges -- are called
  changes.
\end{itemize}

\subsection{Modelling distributiveness}

In the following section we propose an abstract model for distributed
repositories and their interaction.

\subsubsection{The basic notions: Repository, branch and change}

A \emph{repository} $r$ for a certain project is the set of
development branches the project consists of.

A \emph{branch} $b$ is an ordered pair $(\beta, s)$ with $\beta$ being
a \emph{branch identifier} and $s$ being a sequence of changes.

A \emph{change} $C$ is the building block of repositories.  Addition
of changes to a repository is the basic modification of repositories.

A change can be a commit or a merge.

A \emph{commit} is a 4-tuple $(\nu, \beta, \alpha, f)$ with
\begin{itemize}
\item $\nu$ being the branch-unique change-number,
\item $\beta$ being the branch identifier of the branch this commit
  belongs to,
\item $\alpha$ being an access identifier and
\item $f$ being a map $\mathnormal{filename} \rightarrow
  \mathnormal{revision\_number}$.
\end{itemize}

The domain of $f$ is the set of files modified by a certain commit
$c$. $f\ \mathtt{file}$ is the revision-number the backend assigned to
\texttt{file}, which is necessary to retrieve the state of
\texttt{file} at checkin time of $c$.

The \emph{access identifier} $\alpha$ is an abstraction layer for the
backend. This includes things like hostname and path where the
repository is stored, the corresponding authentication method and
functions to checkin and -out files to and from this backend.

A \emph{merge} is a change with two ancestors: One is its predecessing
change in the same branch the merge goes to. The second one is the
change being merged in which is called \emph{origin} of the merge.

The origin of a merge was a change at some point in history to another
branch. Consequently, a merge $m$ is a 5-tuple $(\nu, \beta, \alpha,
f, o)$. The first four elements have the same meaning as in a commit,
the fifth links this change to its origin.

The origin $o$ is a 4-tuple $(\nu, \beta, \alpha, \varrho)$. The
meaning of $\nu$, $\beta$, $\alpha$ is the same as for commits. But in
this case they establish a link to a (possibly remote) pchange with
number $\nu$ on branch $\beta$ reachable via access identifier
$\alpha$.

$\varrho$ indicates if the change-part of a merge $\varrho$eplicates
the commit part of its origin. In this case traversing history along
the merge's branch leads to the same result as traversing the origin's
history -- at least for the one commit.  This feature is useful as
merges are used for inheritance.  $\varrho$ is not necessary for
inheritance to work but is convenient as it indicates that traversing
the origin's history is the same as traversing the merge's history.

changes, merges and commits can be implemented as classes like the
following diagram shows:

\begin{center}
  \setlength{\unitlength}{1mm}
  \begin{picture}(25, 50) \put(0,
    40){\framebox(25,10){\texttt{change}}} \put(12.5, 30){\vector(0,
      1){10}} \put(0, 20){\framebox(25,10){\texttt{commit}}}
    \put(12.5, 10){\vector(0, 1){10}} \put(0, 0){\framebox(25,
      10){\texttt{merge}}}
  \end{picture}
\end{center}


\subsubsection{Properties of the basic notions}

The following helper functions are introduced for brevity:

\begin{itemize}
\item $\mathnormal{nu}:\ \mathnormal{branch} \rightarrow \mathbb{N}$ \\
  The number of a branch (its identifier).
\item $\mathnormal{ma}:\ \mathnormal{branch} \rightarrow
  \mathnormal{filename}$ \\
  Returns the name of the map file for this branch. More on map files
  follows in a later section.
\item $\mathnormal{sy}:\ \mathnormal{branch} \
  \mathnormal{string}$ \\
  A symbolic name for this branch.
\item $\mathnormal{de}:\ \mathnormal{branch}
  \rightarrow \mathnormal{string}$ \\
  A description of this branch.
\item $\mathnormal{nu}:\ \mathnormal{commit} \rightarrow \mathbb{N}$ \\
  The number of a change.
\item $\mathnormal{br}:\ \mathnormal{commit}
  \rightarrow \mathnormal{branch\_identifier}$ \\
  The branch identifier of the branch the change belongs to.
\item $\mathnormal{ac}:\ \mathnormal{commit}
  \rightarrow \mathnormal{access\_identifier}$ \\
  The access identifier of the change.
\item $\mathnormal{fi}:\ \mathnormal{commit} \rightarrow
  (\mathnormal{filename}
  \rightarrow \mathnormal{revision\_number})$ \\
  The filename-revision-map of this change.
\item $\mathnormal{oi}:\ \mathnormal{merge} \rightarrow
  \mathnormal{origin}$ \\
  The origin of the merge.
\item $\mathnormal{nu}:\ \mathnormal{origin} \rightarrow \mathbb{N}$ \\
  The origin's change-number.
\item $\mathnormal{br}:\ \mathnormal{origin} \
  \mathnormal{branch\_identifier}$ \\
  The origin's branch identifier.
\item $\mathnormal{ac}:\ \mathnormal{origin} \rightarrow
  \mathnormal{access\_identifier}$ \\
  How to access the origin (its access identifier).
\item $\mathnormal{re}:\ \mathnormal{origin} \rightarrow \mathbb{B}$ \\
  The replication identifier. If $\mathnormal{re}(o) = \mathtt{true}$
  histories are equal. In general it is $\mathtt{false}$.
\end{itemize}

This abbreviation is used for merges:

$\hat C = (\nu', \beta', \alpha', f', o)$ with $o=(\nu, \beta, \alpha,
d)$ and $C=(\nu, \beta, \alpha, f)$. That means $\hat C$ is the result
of mergeing change $C$ to some other branch $\beta'$.

Now follow important properties for repositories and their elements.

A repository $r$ is a set of branches:

\begin{displaymath}
  r = \{b \ |\ b\ :\ \mathnormal{branch}\}
\end{displaymath}

Branch numbers must be unique:

\begin{displaymath}
  \forall b_1, b_2 \in r:\, \mathnormal{nu}\ b_1 \neq \mathnormal{nu}\ b_2
\end{displaymath}

A branch $b$ is a sequence of changes, which all belong to $b$ and are
accessible in the same way:

\begin{displaymath}
  b\ :\ \mathnormal{change}\ \mathnormal{sequence}\ \wedge\
  \forall C_1, C_2 \in b:\, \mathnormal{br}\ C_1 = \mathnormal{br}\ C_2\ \wedge
  \mathnormal{ac}\ C_1 = \mathnormal{ac}\ C_2
\end{displaymath}

Change numbers in a branch must be unique:

\begin{displaymath}
  \forall C_1, C_2 \in b:\, \mathnormal{nu}\ C_1 \neq
  \mathnormal{nu}\ C_2
\end{displaymath}

\subsubsection{Operations on repositories}

The effect of basic operations is described below. $r$ is an existing
repository and initially looks like this: $r = \{(\beta, \langle C_1,
C_2, C_3, \ldots, C_n\rangle )\}$

The following notations are used:

\begin{itemize}
\item $C \rhd C'$: $c$ is merged in $c'$.
\item $\triangledown C$: A branch from $C$ is created, i.~e. $C \rhd
  \boxempty$ is performed with $\boxempty$ being the empty branch.
\item $\underline\triangledown_n C$: A branch from $C$ is created
  including $n$ predecessors of $C$. $\underline\triangledown_0 C$ is
  equivalent to $\triangledown C$.  An operation
  $\underline\triangledown_n C$ with $n>0$ is called inheriting from
  $C$.  If all predecessors of $C$ are inherited this is denoted by
  $\underline\triangledown_\star$ and called total inheritance and in
  fact is replication. A $\underline\triangledown_n$ operation is the
  only one to cause $\mathnormal{re}(\hat c)\neq \mathtt{false}$. For
  convenience, $\varrho$ is always false if not mentioned in the tuple
  $o$ -- $(\nu, \beta, \alpha, \mathtt{false})=(\nu, \beta, \alpha)$
  -- or indicated otherwise.
\end{itemize}

The basic operations inherit, branch, merge are ``defined'' by the
following transitions:

\begin{itemize}
\item The empty repository $r'=\emptyset$ \emph{inherits} the whole
  repository $r$ ($\underline\triangledown_\star C_n$):

  $r' = \{(\beta', \langle m'_1, m'_2, m'_3, \ldots, m'_n\rangle)\}$
  with $m'_i = (\nu'_i, \beta'_i, \alpha'_i, f'_i, \hat C_i)$ (as
  reminder: $\hat C_i = (\nu_i, \beta_i, \alpha_i, \mathtt{true})\
  \wedge\ C_i=(\nu_i, \beta_i, \alpha_i, f)$, note $\forall\ \hat
  c_i:\, \mathnormal{re}\ d_i=\mathtt{true}$).

\item $r'$ \emph{inherits parts} of the repository
  ($\underline\triangledown_ {n-k}C_k$):

  $r'=\{(\beta', \langle m'_1, m'_2, m'_3, \ldots, m'_{n-k},
  m'_{n-k+1}\rangle)\}$ with $m'_i=(\nu'_i, \beta'_i, \alpha'_i, f,
  \hat C_{i+k-1})$ and $C_{i}=(\nu_i, \beta_i, \alpha_i, f_i)$, $k<n$.

\item Create a \emph{branch} from a change $C_k$ inside $r$
  ($\triangledown C_k$):

  $r = \{(\beta, \langle C_1, \ldots, C_k, \ldots, C_n\rangle),
  (\beta^2, \langle m_1^2 \rangle)\}$ with $m_1^2=(\nu_1^2, \beta_1^2,
  \alpha_1^2, f_1^2, \hat C_k)$ and $\mathnormal{re}\ \hat
  C_k=\mathtt{false}$.

  After doing some changes to the new branch the repository looks like

  $r = \{(\beta, \langle C_1, \ldots, C_k, \ldots, C_n\rangle),
  (\beta^2, \langle m_1^2, C_2^2, C_3^2, \ldots, C_m^2\rangle )\}$

\item Inside $r$, the heads of two branches are \emph{merged} ($C_n^1
  \rhd C_m^2$):

  Before the merge, the repository has this form:

  $\{(\beta^1, \langle C_1^1, C_2^1, \ldots, C_n^1\rangle), (\beta^2,
  \langle C_1^2, C_2^2, \ldots, C_m^2\rangle)\}$

  Afterwards branch $b^2$ has one more change:

  $\{(\beta^1, \langle C_1^1, C_2^1, \ldots, C_n^1\rangle), (\beta^2,
  \langle C_1^2, C_2^2, \ldots, C_m^2, m_{m+1}^2\rangle)\}$

  with $m_{m+1}^2=(\nu_{m+1}^2, \beta_{m+1}^2, \alpha_{m+1}^2,
  f_{m+1}^2, \hat C_n^1)$

  It is obvious that the result looks the same when changes from
  different repositories are merged. Given

  $r = \{(\beta, \langle C_1, C_2, C_3, \ldots, C_n\rangle)\}$ and \\
  $r' = \{(\beta', \langle C'_1, C'_2, C'_3, \ldots, C'_m\rangle)\}$,

  merging their heads $c'_m \rhd C_n$ leads to one more change in $r$:

  $r = \{(\beta, \langle C_1, C_2, C_3, \ldots, C_n,
  m_{n+1}\rangle)\}$ with a link $\hat C'_m$ as last element in
  $m_{n+1}$.
\end{itemize}

\subsubsection{How to store branches}

\textsc{metacvs} introduces F-files. They are called F-files because
their names start with an ``F'' followed by a dash and 32 randomly
generated hexadecimal digits. The name of an F-file is the unique
identifier for a file which is under \textsc{metacvs}'s control.

\textsc{metacvs} introduces a mapping $\mathnormal{F\_file}
\rightarrow filename$ to map the unique identifiers to a ``physical''
filename. This map ist stored in a file called \texttt{MAP} which has
the format:

\begin{verbatim}
(mapping*)
\end{verbatim}

A \texttt{mapping} is a list which stores the F-filename, the
``physical'' filename and meta information like executable state of
the file:

\begin{verbatim}
(:KEY
 "F-filename"
 "filename"
 (meta-information))
\end{verbatim}

\texttt{:KEY} can be \texttt{:FILE} indicating that \texttt{filename}
is an ordinary file or \texttt{:LINK} to version symbolic links.

The main goal of the \texttt{MAP}-file is to resolve the trouble
\textsc{cvs} has with directories and file moves. The entire project
hierarchy is flattened (my guess is that the ``F'' in the F-files
stands for \emph{flat}) and structure is provided by the
\texttt{MAP}-file.

A project $p$ may have a structure like

\begin{displaymath}
  \begin{array}{l}
    p/ \\
    \qquad d_1/ \\
    \qquad\qquad f_{11} \\
    \qquad\qquad f_{12} \\
    \qquad\qquad \vdots \\
    \qquad\qquad f_{1k} \\
    \qquad \vdots \\
    \qquad d_m/ \\
    \qquad\qquad f_{m1} \\
    \qquad\qquad \vdots \\
    \qquad\qquad f_{ml}
  \end{array}
\end{displaymath}

\newpage

This leads to a \texttt{MAP}-file as follows -- the $a$, $b$, $c$,
\ldots are randomly generated numbers:

\begin{displaymath}
  \begin{array}{l@{}l}
    \mathtt{((} & \mathtt{:KEY} \\
    & \mathtt{"MCVS/F-}a \mathtt{"} \\
    & \mathtt{"}d_1/f_{11}\mathtt{")} \\
    \mathtt{(}  & \mathtt{:KEY} \\
    & \mathtt{"MCVS/F-}b \mathtt{"} \\
    & \mathtt{"}d_1/f_{12}\mathtt{")} \\
    & \vdots \\
    \mathtt{(}  & \mathtt{:KEY} \\
    & \mathtt{"MVCS/F-}c \mathtt{"} \\
    & \mathtt{"}d_1/f_{1k}\mathtt{")} \\
    & \vdots \\
    \mathtt{(}  & \mathtt{:KEY} \\
    & \mathtt{"MCVS/F-}d \mathtt{"} \\
    & \mathtt{"}d_m/f_{m1}\mathtt{")} \\
    & \vdots \\
    \mathtt{(}  & \mathtt{:KEY} \\
    & \mathtt{"MCVS/F-}e \mathtt{"} \\
    & \mathtt{"}d_m/f_{ml}\mathtt{"))} \\
  \end{array}
\end{displaymath}

The reason that all F-file paths are prepended by \texttt{MCVS} is
that \textsc{metacvs} stores its administrative information in a
directory \texttt{MCVS} inside the project directory -- like
\texttt{cvs} does with its \texttt{CVS} directory. The \texttt{MCVS}
directory keeps all F-files.

The trick is to hard link the F-files into the project's sandbox
according to the \texttt{MAP}-file. Now moving or renaming files is
done by a \textsc{metacvs} command which renames the hard links and
modifies the mappings. A move conflict appears as a conflict in the
\texttt{MAP}-file.

As gotcha, \texttt{MAP} enables version control of links and
additional bits like the executable bit.

In later sections the \texttt{MAP}-file will be shown as a a map --
omitting the \texttt{MCVS}-path-prefix and quotation marks -- like
this:
\begin{displaymath}
  \mathtt{MAP} =\left[
    \begin{array}{lcl}
      \mathtt{F-}a &\mapsto& d_1/f_{11} \\
      \mathtt{F-}b &\mapsto& d_1/f_{12} \\
      &\vdots & \\
      \mathtt{F-}c &\mapsto& d_1/f_{1k} \\
      \mathtt{F-}d &\mapsto& d_m/f_{m1} \\
      &\vdots & \\
      \mathtt{F-}e &\mapsto& d_m/f_{ml} \\
    \end{array}\right]
\end{displaymath}

The idea to implement branches is to have one \texttt{MAP} file for
each branch to store which files belong to this branch and what their
``physical name'' is. This way, all files are stored without any
hierarchy. There is one constraint: The identifier in the F-file may
not change from branch to branch. If it did, one needed another
mapping from trunk- to branch-F-files. If not done so it would be
impossible to track down the origin of a file in a branch which made
merging impossible.

Hence, it is easiest to let the F-filenames be invariant. As it is
impossible to have two files of the same name in one directory, each
branch gets its own directory. This is no problem because \textsc{cvs}
does not care much about directories and they are completely different
from the project's directories which are stored via the
\texttt{MAP}-file.  At this point, we should track down branch
identifiers to $\mathbb{N}$ such that the branch-directories may have
names like \texttt{1}, \texttt{2}, \texttt{3} etc.

As said, a branch is $(\beta, s)$. To completely store a branch $s$
has to be put somewhere. This will be done in a file $\mathtt{CHANGE}$
which is just a list of changes (merges and commits) stored in
reversed order that the last change is accessible as head of the list.
This file is stored along with the branche's F-files in the
branch-directory.

To complete a repository, a file $\mathtt{BRANCH}$ is installed, which
contains a list of existing branches in reversed order, i.~e. the
branch created last is head of the list.  For each branch, its
identifier $\beta$ ($nu$), a symbolic name ($sy$), a description
($de$) is stored.


\subsection{Case study}

This section discusses in depth a concrete example to demonstrate the
abstract model in a more implementation-close fashion.

\subsubsection{The situation}

Imagine a developer $\mathnormal{Ensel}$ (or a team of developers if
you prefer) writing some compiler in Lisp. Later, others -- may their
name be $\mathnormal{Krete}$ -- get interested in and join in.  They
decide to have their own repository. The interaction of the two
repositories is presented in the following.

\subsubsection{The initial import}

$\mathnormal{Ensel}$ has written a new cool compiler from scratch and
the work done so far manifested in four Lisp-files:

\begin{verbatim}
frontend.lisp
backend.lisp
parser/grammar.lisp
parser/absy.lisp
\end{verbatim}

$\mathnormal{Ensel}$ imports these files into his repository $r$ which
now stores exactly one branch $b$ consisting of one commit $c_1$:

$r=\{(\beta, \langle c_1\rangle)\}$

Digging deeper into this structure brings to light:

\begin{itemize}
\item $(\beta, \langle c_1\rangle)\ :\ \mathnormal{branch}$
  \begin{eqnarray*}
    \mathnormal{sy}\ \beta &=& \mbox{``main''} \\
    \mathnormal{de}\ \beta &=& \mbox{``main line of development''} \\
    \mathnormal{nu}\ \beta &=& 1 \\
  \end{eqnarray*}

\item $c_1\ :\mathnormal{commit}$ and $c_1=(\nu_1, \beta, \alpha,
  f_1)$.
  \begin{eqnarray*}
    \mathnormal{nu}\ c_1 &=& 1 \\
    \mathnormal{br}\ c_1 &=& \beta \\
    \mathnormal{ac}\ c_1 &=& \mathsf{cvs} \\
    \mathnormal{fi}\ c_1 &=& \left[
      \begin{array}{lcl}
        \mathtt{F-}a & \mapsto & \mathsf{r1.1} \\
        \mathtt{F-}b & \mapsto & \mathsf{r1.1} \\
        \mathtt{F-}c & \mapsto & \mathsf{r1.1} \\
        \mathtt{F-}d & \mapsto & \mathsf{r1.1} \\
        \mathtt{MAP} & \mapsto & \mathsf{r1.1} \\
      \end{array}\right] \\
    \mathtt{MAP} &=& \left[
      \begin{array}{lcl}
        \mathtt{F-}a &\mapsto& \mathtt{frontend.lisp} \\
        \mathtt{F-}b &\mapsto& \mathtt{backend.lisp} \\
        \mathtt{F-}c &\mapsto& \mathtt{parser/grammar.lisp} \\
        \mathtt{F-}d &\mapsto& \mathtt{parser/absy.lisp} \\
      \end{array}\right]
  \end{eqnarray*}
\end{itemize}

\subsubsection{$\mathnormal{Krete}$ joins in}

$\mathnormal{Krete}$ is enthusiastic about the new compiler
$\mathnormal{Ensel}$ has written but thinks that some basic
optimization techniques are missing and decides to implement them. For
easier access and to not clutter $\mathnormal{Ensel}$'s repository
with intermediate commits (which likely break the code) a new
repository $r'$ is installed. The first action is
$\underline{\triangledown}_1 c_1$, i.~e. to inherit the last state of
$\mathnormal{Ensel}$'s code (which is the first too at this
time\ldots).

Examination of $r'$ retrieves this information:

\begin{itemize}
\item $r=\{(\beta', \langle m'_1\rangle)\}$
\item $(\beta',\langle m'_1\rangle)\ :\ \mathnormal{branch}$
  \begin{eqnarray*}
    \mathnormal{sy}\ \beta' &=& \mbox{``optimizer''} \\
    \mathnormal{de}\ \beta' &=& \mbox{``includes tail-recursion elimination''} \\
    \mathnormal{nu}\ \beta' &=& 1
  \end{eqnarray*}

\item $m'_1\ :\ \mathnormal{merge}$ and $m'_1=(\nu_1, \beta', \alpha',
  f'_1, o'_1)$.
  \begin{eqnarray*}
    \mathnormal{nu}\ m'_1 &=& 1 \\
    \mathnormal{br}\ m'_1 &=& \beta' \\
    \mathnormal{ac}\ m'_1 &=& \mathsf{cvs} \\
    \mathnormal{fi}\ m'_1 &=& \left[
      \begin{array}{lcl}
        \mathtt{F-}a & \mapsto & \mathsf{r1.1} \\
        \mathtt{F-}b & \mapsto & \mathsf{r1.1} \\
        \mathtt{F-}c & \mapsto & \mathsf{r1.1} \\
        \mathtt{F-}d & \mapsto & \mathsf{r1.1} \\
        \mathtt{MAP} & \mapsto & \mathsf{r1.1} \\
      \end{array}\right] \\
    \mathnormal{oi}\ m'_1 &=& (\nu_1, \beta, \alpha) = \hat c_1 \\
    \mathtt{MAP} &=& \left[
      \begin{array}{lcl}
        \mathtt{F-}a &\mapsto& \mathtt{frontend.lisp} \\
        \mathtt{F-}b &\mapsto& \mathtt{backend.lisp} \\
        \mathtt{F-}c &\mapsto& \mathtt{parser/grammar.lisp} \\
        \mathtt{F-}d &\mapsto& \mathtt{parser/absy.lisp} \\
      \end{array}\right]
  \end{eqnarray*}
\end{itemize}

The optimizer code is partly performed on the Absy and partly on the
backend's intermediate language. Thus, the first commit moves the file
\texttt{absy.lisp} up one level in the directory structure to reflect
that it is manipulated in the frontend as well as in the backend. Lots
of new code is inserted into the backend to implement the optimizer.

This commit is added to $\beta'$: $c'_2=(\nu'_2, \beta', \alpha',
f'_2)$
\begin{eqnarray*}
  \mathnormal{nu}\ c'_2 &=& 2 \\
  \mathnormal{br}\ c'_2 &=& \beta' \\
  \mathnormal{ac}\ c'_2 &=& \mathsf{cvs} \\
  \mathnormal{fi}\ c'_2 &=& \left[
    \begin{array}{lcl}
      \mathtt{F-}b & \mapsto & \mathsf{r1.2} \\
      \mathtt{MAP} & \mapsto & \mathsf{r1.2} \\
    \end{array}\right] \\
  \mathtt{MAP} &=& \left[
    \begin{array}{lcl}
      \mathtt{F-}a &\mapsto& \mathtt{frontend.lisp} \\
      \mathtt{F-}b &\mapsto& \mathtt{backend.lisp} \\
      \mathtt{F-}c &\mapsto& \mathtt{parser/grammar.lisp} \\
      \mathtt{F-}d &\mapsto& \mathtt{absy.lisp} \\
    \end{array}\right]
\end{eqnarray*}

\subsubsection{Feedback}

The performance of the code $\mathnormal{Krete}$'s optimizer puts out
is astonishing and $\mathnormal{Ensel}$ wants to have it in to see the
effect on the language extensions he plans to include. The optimizer
is merged into the first version of the compiler: $c'_2 \rhd c_1$.
This produces merge $m_2=(\nu_2, \beta, \alpha, f_2, o_2)$.
\begin{eqnarray*}
  \mathnormal{nu}\ m_2 &=& 2 \\
  \mathnormal{br}\ m_2 &=& \beta' \\
  \mathnormal{ac}\ m_2 &=& \mathsf{cvs} \\
  \mathnormal{fi}\ m_2 &=& \left[
    \begin{array}{lcl}
      \mathtt{F-}b & \mapsto & \mathsf{r1.2} \\
      \mathtt{MAP} & \mapsto & \mathsf{r1.2} \\
    \end{array}\right] \\
  \mathnormal{oi}\ m_2 &=& (\nu'_2, \beta', \alpha') = \hat c'_2 \\
  \mathtt{MAP} &=& \left[
    \begin{array}{lcl}
      \mathtt{F-}a &\mapsto& \mathtt{frontend.lisp} \\
      \mathtt{F-}b &\mapsto& \mathtt{backend.lisp} \\
      \mathtt{F-}c &\mapsto& \mathtt{parser/grammar.lisp} \\
      \mathtt{F-}d &\mapsto& \mathtt{absy.lisp} \\
    \end{array}\right]
\end{eqnarray*}

\medskip To summarize, the repositories now have the same code and
look like this:
\begin{eqnarray*}
  r  &=& \{(\beta, \langle c_1, m_1\rangle)\} \\
  r' &=& \{(\beta', \langle m'_1, c'_2\langle)\}
\end{eqnarray*}

\subsubsection{Branching ``locally''}

$\mathnormal{Ensel}$ decides that it is probably better to develop the
language extensions in a separate branch: $\triangledown m_2$.

His repository now contains two branches:

$r=\{(b, \langle c_1, m_2\rangle), (\beta^2, \langle m_1^2
\rangle)\}$.

Let's examine this a bit further:
\begin{itemize}
\item $(\beta^2, \langle m_1^2 \rangle\ :\ \mathnormal{branch}$
  \begin{eqnarray*}
    \mathnormal{sy}\ \beta^2 &=& \mbox{``langext''} \\
    \mathnormal{de}\ \beta^2 &=&
    \mbox{``support for sub-types based on set constraints''} \\
    \mathnormal{nu}\ \beta^2 &=& 2
  \end{eqnarray*}

\item $m_1^2\ :\ \mathnormal{merge}$ and $m_1^2=(\nu_1^2, \beta^2,
  \alpha^2, f_1^2, o_1^2)$.
  \begin{eqnarray*}
    \mathnormal{nu}\ m'_1 &=& 1 \\
    \mathnormal{br}\ m'_1 &=& \beta^2 \\
    \mathnormal{ac}\ m'_1 &=& \mathsf{cvs} \\
    \mathnormal{fi}\ m'_1 &=& \left[
      \begin{array}{lcl}
        \mathtt{F-}a & \mapsto & \mathsf{r1.1} \\
        \mathtt{F-}b & \mapsto & \mathsf{r1.1} \\
        \mathtt{F-}c & \mapsto & \mathsf{r1.1} \\
        \mathtt{F-}d & \mapsto & \mathsf{r1.1} \\
        \mathtt{MAP} & \mapsto & \mathsf{r1.1} \\
      \end{array}\right] \\
    \mathnormal{oi}\ m_1^2 &=& (\nu_2, b, \alpha) = \hat m_2 \\
    \mathtt{MAP} &=& \left[
      \begin{array}{lcl}
        \mathtt{F-}a &\mapsto& \mathtt{frontend.lisp} \\
        \mathtt{F-}b &\mapsto& \mathtt{backend.lisp} \\
        \mathtt{F-}c &\mapsto& \mathtt{parser/grammar.lisp} \\
        \mathtt{F-}d &\mapsto& \mathtt{absy.lisp} \\
      \end{array}\right]
  \end{eqnarray*}
\end{itemize}

\subsubsection{Hard work\ldots}

\ldots lets the code base grow and the repositories have changed:
\begin{eqnarray*}
  r &=& \{(\beta, \langle c_1, m_2\rangle), (\beta^2, \langle m_1^2, c_2^2,
  c_3^2, c_4^2\rangle)\} \\
  r' &=& \{(\beta', \langle m'_1, c'_2, c'_3\rangle)\}
\end{eqnarray*}

The most important changes always happen to $f$ and \texttt{MAP}.  As
the next step will be merging the head from $\beta^2$ into $\beta'$ we
look at $f$ and \texttt{MAP} of $c_4^2$:
\begin{eqnarray*}
  \mathnormal{fi}\ c_4^2 &=& \left[
    \begin{array}{lcl}
      \mathtt{F-}a & \mapsto & \mathsf{r1.1} \\
      \mathtt{F-}b & \mapsto & \mathsf{r1.1} \\
      \mathtt{F-}c & \mapsto & \mathsf{r1.1} \\
      \mathtt{F-}d & \mapsto & \mathsf{r1.2} \\
      \mathtt{F-}e & \mapsto & \mathsf{r1.2} \\
      \mathtt{F-}f & \mapsto & \mathsf{r1.3} \\
      \mathtt{MAP} & \mapsto & \mathsf{r1.2} \\
    \end{array}\right] \\
  \mathtt{MAP} &=& \left[
    \begin{array}{lcl}
      \mathtt{F-}a &\mapsto& \mathtt{frontend.lisp} \\
      \mathtt{F-}b &\mapsto& \mathtt{backend.lisp} \\
      \mathtt{F-}c &\mapsto& \mathtt{parser/grammar.lisp} \\
      \mathtt{F-}d &\mapsto& \mathtt{absy.lisp} \\
      \mathtt{F-}e &\mapsto& \mathtt{constraints/search.lisp} \\
      \mathtt{F-}f &\mapsto& \mathtt{constraints/propagate.lisp} \\
    \end{array}\right]
\end{eqnarray*}

Now it is time for the optimizer team to merge in the constraint
solving code to write some optimization for it: $c_4^2 \rhd c'_3$.

The result is $m'_4=(\nu'_4, \beta', \alpha', f'_4, o'_4)$ with the
following maps:
\begin{eqnarray*}
  \mathnormal{fi}\ m_4 &=& \left[
    \begin{array}{lcl}
      \mathtt{F-}a & \mapsto & \mathsf{r1.1} \\
      \mathtt{F-}b & \mapsto & \mathsf{r1.2} \\
      \mathtt{F-}c & \mapsto & \mathsf{r1.1} \\
      \mathtt{F-}d & \mapsto & \mathsf{r1.2} \\
      \mathtt{F-}e & \mapsto & \mathsf{r1.1} \\
      \mathtt{F-}f & \mapsto & \mathsf{r1.1} \\
      \mathtt{MAP} & \mapsto & \mathsf{r1.3} \\
    \end{array}\right] \\
  \mathtt{MAP} &=& \left[
    \begin{array}{lcl}
      \mathtt{F-}a &\mapsto& \mathtt{frontend.lisp} \\
      \mathtt{F-}b &\mapsto& \mathtt{backend.lisp} \\
      \mathtt{F-}c &\mapsto& \mathtt{parser/grammar.lisp} \\
      \mathtt{F-}d &\mapsto& \mathtt{absy.lisp} \\
      \mathtt{F-}e &\mapsto& \mathtt{constraints/search.lisp} \\
      \mathtt{F-}f &\mapsto& \mathtt{constraints/propagate.lisp} \\
    \end{array}\right]
\end{eqnarray*}


\subsection{Repository actions \label{ops}}

This section describes the basic actions branch, checkin, merge,
update in detail.

\subsubsection{Sandbox and Repository}

The sandbox contains the checked out copy of a certain change of a
certain branch. The information which branch and commit lives in the
sandbox is stored in the administrative directory.

Two kinds of operations have to be distinguished:
\begin{itemize}
\item \emph{Repository only} operations which do neither necessarily
  require any sandbox nor do changes in a sandbox effect the result of
  an operation.  Branching and merging are such operations.
\item \emph{Sandbox} operations may cause changes in the repository
  but not necessarily have to but they either change the sandbox or
  changes in the sandbox effect the repository. Update, checkin belong
  to this category.
\end{itemize}


\subsubsection{commit}

A commit adds a new commit to a branch. Checking in is a several step
procedure:

\begin{enumerate}
\item Get the latest $\mathtt{CHANGE}$.
\item Comparison of the latest $\mathtt{CHANGE}$ with the one present
  in the sandbox gives the files changed in the meantime from other
  commits.
\item Check which files are modified locally.
\item The intersection of the two file sets gives potential conflicts.
  If this intersection is empty create a new commit and prepend it to
  the current $\mathtt{CHANGE}$ The file map for the new commit is
  generated out of the revision numbers the backend assigns to each
  file on checkin.

  Check in the modified $\mathtt{CHANGE}$ and throw away the old one.
\item If the intersection is not empty check if potential conflicts
  are real conflicts. That means to merge to last repository version
  of all possibly conflicting files with the one in the working copy
  temporarily. If none of the merges fails a new commit is added as
  indicated above.
\item If the merging fails for at least one file tell the user about
  it and mark the conflicing parts. No file will be checked in and the
  $\mathtt{CHANGE}$ file is reverted to the original one.
\end{enumerate}


\subsubsection{branch}

Branching is a repository only operation but some temporary hard disk
space will be necessary.

As branching cannot conflict it is a pretty straight forward
procedure:
\begin{enumerate}
\item Get the latest $\mathtt{BRANCH}$ file from the repository.
\item Add a new branch with the appropriate information (branch
  identifier, name, description etc.)
\item Create a new \texttt{CHANGE}-file for this branch.
\item Get the the \texttt{CHANGE}-file of the branch which is branched
  from and extract the change which is head of the new branch.
\item Retrieve all files revisions indicated in this changes file map.
\item Check in the retrieved files and create a corresponding merge
  with modified file map and the branch's origin.
\item Check in the \texttt{CHANGE}- and modified
  \texttt{BRANCH}-files.
\end{enumerate}

If replication takes places the steps 4 to 8 are repeated for all
changes to be replicated starting with the oldest one. For all changes
incorporated $\varrho$ will hold.

\subsubsection{merge}
Merging is -- like branching -- a repository only operation.

Merging is essentially a checkin of a certain change to a certain
branch without a sandbox and creating a merge instead of a commit:

\begin{enumerate}
\item Put the change to merge into a temporary sandbox.
\item Get the \texttt{CHANGE}-file from the change to which the merge
  will be applied.
\item In the case of a merge, conflicts cannot in general be uncovered
  by comparing the \texttt{CHANGE}-files. Instead dry-run merges have
  to be done. If all goes well a new merge is created and prepended to
  the \texttt{CHANGE}-file and all files including the
  \texttt{MAP}-file are checked in.
\item If conflicts arise the user has to resolve them in the temporary
  sandbox and the merge has to be restarted.
\end{enumerate}

\subsubsection{update}

The update operation is a sandbox only operation but may cause
conflicts.

\begin{enumerate}
\item Get the latest \texttt{CHANGE}-file for the sandbox' branch and
  compare it to the \texttt{CHANGE}-file in the sandbox.
\item If changes were made to the branch check out the changed files
  and merge them into the files in the snadbox. Conflicts are marked
  and have to be handled by the user.
\end{enumerate}


% \subsection{Access identifier -- backend abstraction}

% As a first approximation for \textsc{cvs} as the only backend it is
% sufficient to store a function in $\alpha$ which is able to set the
% environment-variable \texttt{CVSROOT} (and perhaps
% \texttt{CVS\_RSH}). This function is called before \textsc{cvs} is
% invoked to set up the necessary environment.  All those functions
% should be stored in a table (a file called \texttt{ACCESSOR}) with
% increasing numbers such that access identifiers in changes contain
% the appropriate number.

% If a repository is moved on another machine or something the like
% the function is exchanged and a new revision of \texttt{ACCESSOR} is
% stored (this way even movements of repositories are recorded).


\begin{thebibliography}{Loe99}
\bibitem{HMI05}[HMI05] Yasuhiro Hayase / Makoto Matsushita / Katsuro
  Inoue; \emph{Revision control system using delta script of syntax
    tree}; ACM Press 200
\bibitem{MCVS04}[MCVS04] Kaz Kylheku; \texttt{metacvs};
  \texttt{http://users.footprints.net/\textasciitilde{}kaz/mcvs.html}
\bibitem{PS05}[PS05] Peter Seibel; \emph{Practical Common Lis}; \texttt{http://www.gigamonkeys.com/book/} is a very good
  \textsc{Common Lisp} book.
\bibitem \texttt{http://clisp.cons.org/} is the home of \textsc{clisp}.
\bibitem \texttt{http://common-lisp.net/project/slime/} is the home of
  \textsc{slime}, a good interface to different \textsc{Common~Lisp}
  systems for \textsc{emacs}.
\bibitem \texttt{http://wiki.gnuarch.org/Signing\_20Archives} is the
  Howto of the GNU arch project using signed archives.
\bibitem \texttt{http://www.gnupg.org} GNU Privacy Guard.
\bibitem \textsc{hocvs} is an extension to \textsc{metacvs} with change
  based revisions. It is done with \textsc{cvs} tags, I believe, but
  perhaps worth a glance. It can be found at
  \texttt{http://retes.hu/\textasciitilde{}mega/ho-cvs/}.
\bibitem \textsc{dcvs} is a combination of \textsc{cvs} and
  \textsc{cvsup} to support change sets and distributed repositories.
  It is found at \texttt{http://www.elegosoft.com/dcvs/}.

\end{thebibliography}

The links below should be in included into the bibliography and be
properly referenced in the text. There are references missing I think.

\begin{itemize}
\item
  \texttt{http://users.footprints.net/\textasciitilde{}kaz/mcvs.html}
  is the home of \textsc{metacvs}.
\item \texttt{http://www.gigamonkeys.com/book/} is a very good
  \textsc{Common Lisp} book.
\item \texttt{http://clisp.cons.org/} is the home of \textsc{clisp}.
\item \texttt{http://common-lisp.net/project/slime/} is the home of
  \textsc{slime}, a good interface to different \textsc{Common~Lisp}
  systems for \textsc{emacs}.
\item \texttt{http://wiki.gnuarch.org/Signing\_20Archives} is the
  Howto of the GNU arch project using signed archives.
\item \texttt{http://www.gnupg.org} GNU Privacy Guard.
\item \textsc{hocvs} is an extension to \textsc{metacvs} with change
  based revisions. It is done with \textsc{cvs} tags, I believe, but
  perhaps worth a glance. It can be found at
  \texttt{http://retes.hu/\textasciitilde{}mega/ho-cvs/}.
\item \textsc{dcvs} is a combination of \textsc{cvs} and
  \textsc{cvsup} to support change sets and distributed repositories.
  It is found at \texttt{http://www.elegosoft.com/dcvs/}.
\end{itemize}


% I think, all stuff below can basically go away.
%
%
%
% \section{Comments on other software}

% \subsection{\textsc{dcvs}}

% \textsc{dcvs} implements distributed repositories. But it assigns
% responsibilities for certain parts (branches) of the code to a
% particular site which is not so flexible and not what we want, I
% think.  They use \textsc{cvsup} to synchronize repositories which
% might be interesting because the problem of how to exchange data
% without checking out all necessary revisions and not touching the
% repository directly is not yet solved.
 
% Anyway, \textsc{dcvs} implements a concept of change sets and
% snapshots.  Snapshots a hooks into the code at certain point in time
% and change sets are more or less diffs between to snapshots (what
% others call patchsets).  In our termninology, the snapshot is the
% change set and a patchset may contain diffs between two or more
% change sets.


% \section{Todo and milestones}

% \subsection{Zeroth milestone}
% Specification for changes. This includes the handling of branches
% and merges.  \ldots and is finished by now.

% \subsection{First milestone}
% The first thing to do is implementing change and branch handling as
% describes in section \ref{distrib} and writing an interface to the
% \textsc{gpgme} library. At this point, we have to decide what
% \textsc{Common Lisp} implementation will be used because calls to
% foreign functions is not included in the ANSI standard.
% Documentation for \textsc{clisp} can be found at
% \texttt{http://clisp.sourceforge.net/impnotes.html\#dffi} and the
% like for \textsc{cmucl} at
% \texttt{http://common-lisp.net/project/cmucl/doc/cmu-user/aliens.html}.
% The decision has been taken and \textsc{sbcl} will be the
% implementation of choice. For all FFI-stuff \textsc{UFFI} should be
% taken and proper \textsc{ASDF} definition be written.

% \subsection{Second milestone}
% Specification and implementation of the code signing features.

% \subsection{Third milestone}
% Optional things:
% \begin{itemize}
% \item Convenient wrappers for the most common use cases based on the
%   basic operations described in section \ref{ops} like branching and
%   haveing the branch as sandbox and the same for merging etc.
% \item Integration of other backends than \textsc{cvs}.
% \item Makeing the code compatible (especially buildable!) with other
%   \textsc{Common Lisp} implementations.
% \item \textsc{emacs} integration.
% \item Something else from the \emph{nice features to have section}.
% \end{itemize}

% \subsection{Other things}
% \begin{itemize}
% \item Switching the whole thing to use \textsc{asdf} would be great.
%   Furthermore, the necessity to build a Lisp image with some Unix
%   bindings to build the system should be removed. I think,
%   \textsc{clisp}'s \texttt{POSIX}-package could be sufficient.
% \item \textsc{metacvs} does not support empty directories at the
%   moment.  We could patch this.
% \item The behaviour when issuing a \textsc{cvs} command not shadowed
%   by \textsc{metacvs} is to fire the debugger. Perhaps it is
%   sensible to just pass those commands to the \textsc{cvs} binary
%   and end up in the debugger only if the command is unknown to
%   \textsc{cvs} too.  (I don't think we should do this. Invoking
%   \textsc[cvs] directly is somewhat dangerous and all relevant calls
%   should be wrapped [Florian]).
% \item The user should not see any F-files. Especially not in
%   expanded keywords like \texttt{\$Id\$} and the commit message.
% \end{itemize}

% \subsection{Nice features to have}

% \begin{itemize}
% \item An operation to gather all origins with $\neg \varrho$ as
%   local copies into a repository (ending the distributiveness,
%   perhaps for a restart).
% \item A possibility to generate a graph to show the links between
%   repositories.
% \end{itemize}


\end{document}
