% \documentclass[german, handout]{beamer}
\documentclass[german, presentation]{beamer}

\newcommand{\GENNF}{\textsc{gennf}}

\usepackage{babel}
\usepackage{stmaryrd}
%\usepackage{pifont}


\usetheme{Rochester}
\usefonttheme{serif}
\usecolortheme[RGB={120,0,16}]{structure}
\setbeamertemplate{navigation symbols}{}

\setbeamertemplate{footline}{
  \setbeamercolor{footer}{fg=white,bg=structure}
  \begin{beamercolorbox}{footer}
    \vskip2pt
    \hspace{1em}Infrastrukturen zur Open Source Softwareentwicklung\hfill
    \insertframenumber\,/\,\inserttotalframenumber\hspace{1em}
    \vskip2pt
  \end{beamercolorbox}
}

\author{Fabian~Otto~{\scriptsize (\texttt{sigsegv@cs.tu-berlin.de})}
  \and Florian~Lorenzen~{\scriptsize (\texttt{florenz@cs.tu-berlin.de})}}

\title{\GENNF}
\subtitle{Verteiltes Versionsmanagement}
\institute{TU~Berlin, FG Formale Modelle, Logik und Programmierung, \\
Infrastrukturen zur Open Source Softwareentwicklung WS05/06 \\
Bernd Mahr, Steffen Evers}
\date{11. April 2006}



\begin{document}


\frame[plain]{\titlepage}

\begin{frame}
  \frametitle{"Ubersicht}
  \tableofcontents
\end{frame}


\section{Projekt}


\begin{frame} \frametitle{Ziele}
  \begin{beamercolorbox}{block body}
    \begin{itemize}
    \item Erweiterung von \textsc{metacvs}
    \item Verteilung
      \begin{itemize}
      \item Nutzung mehrerer Repositories
      \item Einfache Zusammenf"uhrung von Entwicklungslinien
      \item \textsc{cvs} als Infrastruktur
      \end{itemize}
    \item Code Signierung
      \begin{itemize}
      \item Sicherstellung der Codequellen
      \item Signierung gesichteter Codezeilen
      \end{itemize}
    \end{itemize}
  \end{beamercolorbox}
\end{frame}


\begin{frame} \frametitle{Erreichtes}
  \begin{beamercolorbox}{block body}
    \begin{itemize}
    \item Verteilung
    \item Eigenst"andiges Frontend f"ur \textsc{cvs}
    \item Kommandozeilen-Programm
    \item Portables \textsc{common~lisp} Projekt
    \item Nebenprodukt: portable Pfadnamenbibliothek
    \end{itemize}
  \end{beamercolorbox}
\end{frame}


\begin{frame} \frametitle{Ablauf}
  \begin{block}{1. H"alfte}
    \begin{description}
    \item[November] Themensuche und Planung
    \item[Dezember] Planung, Portierung von \textsc{metacvs} nach \textsc{sbcl}
    \item[Januar] Neuanfang: Eigenentwurf in \textsc{sbcl}
    \item[Februar] Basisfunktionalit"at Verteilung
    \end{description}
  \end{block}
  \begin{block}{2. H"alfte}
    \begin{description}
    \item[Februar] Verzeichnisstruktur
    \item[M"arz] Frontend, Verzeichnisstruktur
    \item[April] Frontend, Tests, Schliff, Release v0.1
    \end{description}
  \end{block}
\end{frame}


\begin{frame} \frametitle{Probleme I}
  \begin{block}{Recherche}
    \begin{itemize}
    \item Erweiterung von \textsc{metacvs}
      \begin{itemize}
      \item Aufgrund der \textsc{metacvs} Struktur nicht m"oglich
      \item Portierung nach \textsc{sbcl} unn"otiger Zeitverlust
      \item \textsc{metacvs} Code nicht gr"undlich genug inspiziert
      \end{itemize}
    \item Vorhandene verteilte Systeme
      \begin{itemize}
        \item \textsc{bzr} (\textsc{gnu~arch}): vergleichbare Funktionalit"at
%       \item \textsc{darcs}
%       \item \textsc{mercurial}
%       \item \textsc{bzr} (\textsc{gnu~arch})
      \end{itemize}
    \end{itemize}
  \end{block}
  \begin{block}{Mitgliederanzahlverringerung}
    \begin{itemize}
    \item Streichung der Code Signierung
    \end{itemize}
  \end{block}
\end{frame}


\begin{frame} \frametitle{Probleme II}
  \begin{block}{\textsc{cvs}}
    \begin{itemize}
    \item Zuviele unben"otigte Features
    \item Kontrolle der Operationen nicht ausreichend feingranular
    \item Merge f"ur unsere Zwecke ungeeignet
    \item Entwicklungsmodell mit unserem schlecht vertr"aglich
    \end{itemize}
  \end{block}
  \begin{block}{Konsequenzen}
    \begin{itemize}
    \item H"aufige Aufrufe von \textsc{cvs}
    \item Starke Nutzung tempor"aren Plattenspeichers
    \item Einsatz von \textsc{cl-difflib} und eigenen Merge-Routinen
    \end{itemize}
  \end{block}
\end{frame}


\begin{frame} \frametitle{Ergebnis}
  \begin{block}{Verteiltes Versionsmanagement System}
    \begin{itemize}
    \item Basisfunktionalit"at:
      \begin{itemize}
      \item \texttt{add}, \texttt{move}, \texttt{delete}
      \item \texttt{checkout}, \texttt{update}, \texttt{commit}
      \item \texttt{branch}, \texttt{merge}
      \end{itemize}
    \item Nachteile:
      \begin{itemize}
      \item Langsam
      \item Hohe Netzwerklast
      \item Fehlende Kommandos: \texttt{diff}, \texttt{patch},
        \texttt{log}, \ldots
      \end{itemize}
    \end{itemize}
  \end{block}
\end{frame}


\begin{frame}
 \frametitle{\tt (make-instance 'data-structure)}
 \begin{block}{Verteilung} 
   \begin{description}
   \item[\sf Repository] Menge von Branches
   \item[\sf Branch] Sequenz von Changes
   \item[\sf Change] Commit oder Merge
   \item[\sf Merge] Zusammenf"uhrung von Branches
   \item[\sf Commit] Beitrag, "Anderung
   \end{description}
 \end{block}
 \begin{block}{Verzeichnisstruktur} 
   \begin{description}
   \item[\sf 210987] $\mapsto$ \tt parser.lisp
   \item[\sf 123456] $\mapsto$ \tt tests/momo.lisp
   \item[\sf 654321] $\mapsto$ \tt doc/folien.tex
   \end{description}
 \end{block}
\end{frame}



\section{Demonstration}


\begin{frame}[fragile]
  \frametitle{\GENNF{} in Aktion}
  \begin{beamercolorbox}{block body}
    {\small
\begin{semiverbatim}
 $ gennf setup find --root $REPO1 --symbolic-name find
     --description Finite-domain-constraint-solver \vspace{5pt}
 $ gennf checkout find --root $REPO1 \vspace{5pt}
 $ gennf add README packages.lisp fd.asd fd.lisp
     tests.lisp sudoku.lisp send-more-money.lisp \vspace{5pt}
 $ gennf commit
\end{semiverbatim}
    }
  \end{beamercolorbox}
  
  \bigskip

  \begin{center}
    {\Large\bf \alert{Fortsetzung folgt!}}
  \end{center}

\end{frame}

\begin{frame}
  \frametitle{Links}
  \begin{beamercolorbox}{block body}
    \begin{itemize}
    \item Projektseite und -bericht \\ \texttt{http://gennf.berlios.de/}
    \item \GENNF{} v0.1 und \textsc{port-path} v0.1 \\
      \texttt{http://developer.berlios.de/projects/gennf/}
    \item \textsc{sbcl} v0.9.10 \\
      \texttt{http://sbcl.sourceforge.net/}
    \item \textsc{osicat} v0.5.0 \\
      \texttt{http://common-lisp.net/project/osicat/}
    \item \textsc{uffi} v1.5.9 \\
      \texttt{http://uffi.b9.com/}
    \item \textsc{cl-difflib} v0.1 \\
      \texttt{http://www.cliki.net/CL-DIFFLIB}
    \end{itemize}

    
  \end{beamercolorbox}
\end{frame}


\end{document}
